\documentclass[a4paper,11pt]{book}
\setlength{\topmargin} {1 cm}
%\setlength{\oddsidemargin} {1 cm}
%\setlength{\evensidemargin} {1 cm}
\setlength{\footskip} {0 cm}
\usepackage [T1]{fontenc}
\usepackage [utf8]{inputenc}
\usepackage [italian]{babel}
\usepackage {graphicx}
\usepackage {frontespizio}
\usepackage[a4paper, total={6in, 9in}]{geometry}


\begin{document}


\begin{frontespizio}
\Logo{logo.jpg}
\Universita{Verona}
\Facolta{Scienze e Ingegneria}
\Corso[Laurea]{Informatica}
\Titolo{WhatsApp Messenger}
\Relatore{Prof. Damiano Carra}
\Candidato[VR359129]{Alberto Marini}
\Annoaccademico{2013 - 2014}
\end{frontespizio}


\tableofcontents

\chapter{Introduzione}


\chapter{Strumenti utilizzati}

\section{WhatsApp Messenger}
\subsection{Cos'\`e}
WhatsApp Messenger \`e un'applicazione di messaggistica mobile multi piattaforma che consente di scambiarsi messaggi coi propri contatti senza dover pagare gli SMS. WhatsApp Messenger \`e disponibile per iPhone, BlackBerry, Android, Windows Phone e Nokia. Tutti questi telefoni possono scambiarsi messaggi gli uni gli altri. Dato che WhatsApp Messenger si serve dello stesso piano dati Internet usato per le e-mail e la navigazione web, non vi sono costi aggiuntivi per mandare messaggi e restare in contatto coi propri amici ed \`e sicuramente questo uno dei motivi per i quali questa applicazione ha ottenuto un cos\`i gran successo in poco tempo.

Oltre alla messaggistica di base gli utenti di WhatsApp possono creare gruppi, scambiarsi messaggi illimitati, video e messaggi audio multimediali.

L'11 aprile 2014 \`e arrivato il via libera all'acquisizione di WhatsApp da parte di Facebook dalla Federal Trade Commission (Ftc), l'ente governativo americano per la protezione dei consumatori.

\begin{figure}[!ht]
\centering
\includegraphics[scale = 0.3]{Whatsapp.png}
\caption{Interfaccia di WhatsApp}
\end{figure}

\clearpage
%~

\subsection{Come funziona}
Servendosi della rete cellulare, WhatsApp messenger permette di inviare messaggi a qualsiasi altro utente connesso ad una rete. Questa modalit\`a di funzionamento ci permette di affermare che, sicuramente, all'invio di un messaggio viene contattato un server il quale avr\`a il compito di smistarlo al dispositivo di destinazione.

In particolare, all'invio di un messaggio vengono effettuate le seguenti operazioni:
\begin{itemize}
 \item Il messaggio arriva ad un server
 \item Il server comunica al mittente l'avvenuta ricezione del messaggio
 \item Il server inoltra il messaggio al destinatario
 \item Il destinatario comunica al server l'avvenuta ricezione del messaggio
 \item Il server comunica al mittente l'avvenuta ricezione del messaggio da parte del destinatario
\end{itemize}

Quando il server riceve il messaggio, nel dispositivo mittente compare una spunta; quando il destinatario riceve il messaggio al mittente compare la seconda spunta.

La \figurename ~\ref{fig:WhatsApp} illustra tale funzionamento.
%La \figurename~\ref{fig:rete} illustra tale funzionamento.
~
\begin{figure}[h!t]
\centering
\includegraphics[scale = 0.6]{rete}
\caption{Funzionamento WhatsApp}
\label{fig:WhatsApp}
\end{figure}

\clearpage

\section{Shark}
\subsection{Cos'\`e}

\begin{figure}[h!t]
\centering
\includegraphics[scale = 0.6]{shark}
\caption{Anteprima Shark}
\end{figure}

\begin{figure}[h!t]
\centering
\includegraphics[scale = 0.6]{shark_go}
\caption{Anteprima Shark - Running}
\end{figure}

\clearpage

\section{Wireshark}

\subsection{Cos'\`e}
Wireshark \`e un analizzatore di rete. Consente di catturare direttamente i dati da una rete attiva oppure di analizzare file contenenti pacchetti precedentemente ottenuti. Inizialmente, il formato dei file catturati da Wireshark era il formato libpcap, che \`e il formato usato da tcpdump ed altri tools.

~

I pacchetti catturati sono conformi alla libreria pcap. \`E possibile applicare filtri ai pacchetti ottenuti, selezionando, per esempio, solo quelli provenienti da un determinato IP sorgente. I filtri applicabili ai pacchetti seguono le regole della libreria pcap.

L'interfaccia grafica di Wireshark (\figurename ~\ref{fig:Whireshark}) mostra il numero di pacchetti catturati, il tempo trascorso tra la cattura dei pacchetti, l'indirizzo sorgente e quello di destinazione, il protocollo usato, la lunghezza del pacchetto ed altre informazioni.

~

\begin{figure}[!ht]
\centering
\includegraphics[scale = 0.5]{Whireshark}
\caption{Interfaccia di Wireshark}
\label{fig:Whireshark}
\end{figure}

\clearpage

\section{Whois}
Whois \`e un servizio utilizzabile da shell di Ubuntu che permette di visualizzare informazioni riguardanti un determinato indirizzo IP.
In particolare, applicando whois ad un indirizzo IP, vengono visualizzati il nome della rete, il range di indirizzi ai quali la rete fa riferimento, il luogo in cui si situa l'IP ricercato, l'organizzazione che lo gestisce ed altre informazioni di rete.

~

\begin{figure}[!ht]
\centering
\includegraphics[scale = 0.7]{Whois.png}
\caption{Esempio Whois}
\end{figure}

\clearpage

\section{Cloud Monitor}
Cloud Monitor \`e un'azienda leader nel settore del monitoraggio delle prestazioni di siti ed applicazioni Web. Verifica le prestazioni di siti e server grazie a 95 stazioni di monitoraggio disposte in 48 paesi del mondo. 
Dato un indirizzo IP o un sito web, effettua, attraverso le 95 stazioni, ping verso quell'indirizzo registrando l'esito dello stesso e, in caso di ping eseguito con successo, RTT minimo, RTT medio ed RTT massimo (RTT - Round Trip Time, tempo impiegato da un pacchetto di dimensione trascurabile per viaggiare da un computer ad un altro e tornare indietro). 

~

\begin{figure}[!ht]
\centering
\includegraphics[scale = 0.7]{Cloud_Monitor.png}
\caption{Esempio Cloud Monitor}
\end{figure}


\chapter{Misurazioni}

L'obiettivo di questo progetto era di scoprire informazioni riguardo WhatsApp e, in particolare, la modalità di scambio dei messaggi e il dislocamento dei server nel mondo.
Per fare ci\`o, sono state fatte rilevazioni giornaliere per pi\`u di 30 giorni, con l'utilizzo degli strumenti citati nel capitolo precedente.

Attraverso ''Shark'', ogni giorno sono state rilevate le informazioni contenute nei pacchetti scambiati tra 2 dispositivi durante l'invio e la ricezione di messaggi.
Dopodich\`e, sono stati analizzati gli indirizzi IP di destinazione in modo da risalire agli indirizzi dei server di WhatsApp.

Dopo aver scartato gli indirizzi IP di servizi noti (e.g. Facebook, Google, Yahoo), è stato ottenuto un pool di indirizzi associabile all'applicazione studiata.

~

La \figurename ~\ref{fig:rilevazioni} mostra tutti gli indirizzi IP catturati giorno per giorno.  

~

\begin{figure}[!ht]
\centering
\includegraphics[scale = 0.7]{rilevazioni_30gg.png}
\caption{Rilevamenti 30 giorni}
\label{fig:rilevazioni}
\end{figure}

~

Ottenuto questo insieme di indirizzi IP, \`e stato utilizzato il servizio ''Whois'' di Ubuntu per controllare la provenienza di tutti gli indirizzi e l'azienda in possesso degli stessi.
\`E emerso che, alla fine, tutti gli indirizzi trovati fanno parte di 5 range di indirizzi i quali appartengono a due aziende. Le aziende in questione sono la ''SoftLayer'' e la ''ThePlanet''.
''SoftLayer'' \`e una societ\`a del gruppo IBM, \`e stata fondata nel 2005 e ha sede a Dallas, Texas. L'azienda ha acquisito ''ThePlanet'' con sede a Houston, Texas. 
Possiamo quindi affermare che tutti gli indirizzi IP trovati appartengono alla stessa azienda, la ''SoftLayer''.

~

%La Tabella \ref{table:ip} mostra i 5 range di indirizzi trovati e l'azienda che li gestisce.

~

%\begin{table}
%\label{table:ip}
\begin{tabular}{|l|c|c|r|}
\hline
\multicolumn{1}{|c|}{}\\
\multicolumn{1}{|c|}{\textbf{IP - range}} & {\textbf{Company}} & {\textbf{Position}}\\
\multicolumn{1}{|c|}{}\\
\hline
108.168.128.0 - 108.168.255.255 &  SoftLayer Technologies Inc. & Dallas\\
173.192.0.0 - 173.193.255.255 &  SoftLayer Technologies Inc. & Dallas\\
184.172.0.0 - 184.173.255.255 &  ThePlanet.com Internet Services, Inc. & Houston\\
208.43.0.0 - 208.43.255.255 &  SoftLayer Technologies Inc. & Dallas\\
50.22.0.0 - 50.23.255.255 &  SoftLayer Technologies Inc. & Dallas\\
\hline
\caption{Range di IP collegati a WhatsApp}
\end{tabular}
%\end{table}

~

Gli IP trovati, dunque, sono stati associati ad aziende di Dallas e Houston. 
Per analizzare questo fatto e, soprattutto, per cercare di accertare tale posizione, \`e stato utilizzato il servizio di ''Cloud Monitor'', inserendo nel campo di ricerca un indirizzo IP appartenente ad ogni range e controllando l'RTT medio. 
Generalmente, se un terminale si trovasse in America, l'RTT medio proveniente da stati americani verso quel dispositivo sarebbe inferiore rispetto all'RTT medio proveniente da stati europei.

~

%Attraverso il servizio di ''Cloud Monitor'', dunque, sono stati fatti questi test e la Tabella \ref{table:Monitoring} ne riporta i risultati.

~

%\begin{table}
%\label{table:Monitoring}
\begin{tabular}{|l|c|c|r|}
\hline
\multicolumn{1}{|c|}{}\\
\multicolumn{1}{|c|}{\textbf{Punto di controllo}} & {\textbf{Risultato}} & {\textbf{RTT minimo	} & {\textbf{RTT medio} & {\textbf{RTT massimo}\\
\multicolumn{1}{|c|}{}\\
\hline
 & & 108.168.128.0 - 108.168.255.255 &  SoftLayer Technologies Inc. & Dallas\\
 & & 173.192.0.0 - 173.193.255.255 &  SoftLayer Technologies Inc. & Dallas\\
 & & 184.172.0.0 - 184.173.255.255 &  ThePlanet.com Internet Services, Inc. & Houston\\
 & & 208.43.0.0 - 208.43.255.255 &  SoftLayer Technologies Inc. & Dallas\\
 & & 50.22.0.0 - 50.23.255.255 &  SoftLayer Technologies Inc. & Dallas\\
\hline
\caption{Cloud Monitoring - A IP for every range}
\end{tabular}
%\end{table}

~

Come si pu\`o notare, l'RTT medio proveniente da server americani è inferiore rispetto ad altri server. Questa soluzione permette di confermare l'effettiva collocazione geografica degli indirizzi IP trovati.

~

Un altro test effettuato \`e stato quello di effettuare un pre filtro dei pacchetti rilevati dall'applicazione ''Shark''. In questo modo, l'applicazione ha catturato solo i pacchetti aventi come destinazione un indirizzo IP appartenente ad uno dei range scoperti in precedenza.
In questo modo, la quantit\`a di informazione catturata era limitata a quella che ci interessava.
Durante la cattura, durata alcune ore, sono state ottenute altre importanti informazioni. Durante le ore, infatti, il dispositivo cellulare ha cambiato più volte rete (passando da Wi-fi a 3G e viceversa) e, in concomitanza con questi combi, sono cambiati anche gli indirizzi di destinazione.
Questo fatto ci permette di affermare che uno smartphone comunica con un server (deciso al momento dell'inizio della connessione) e cambia con il cambiamento della rete utilizzata dal dispositivo.

~

Come ultimo test effettuato, abbiamo cercato di capire se la \figurename ~\ref{fig:WhatsApp} rispecchiasse il corretto funzionamento dell'applicazione.
Tale figura mostra come due dispositivi si scambiano messaggi passando attraverso un server comune. 
Il controllo effettuato si \`e basato sull'utilizzo contemporaneo di ''Shark'' da parte di due dispositivi per catturare i pacchetti scambiati tra di essi. 
Facendo cos\`i, si \`e potuto controllare l'indirizzo IP di destinazione di entrambi gli smartphone. Se gli IP fossero stati uguali, allora si poteva affermare che, quando due dispositivi comunicano tra di loro, si connettono allo stesso server.
In realt\`a, gli IP di destinazione erano differenti. Questo ci indica con certezza che durante la comunicazione tra pi\`u dispositivi, ogni dispositivo si connette ad un proprio server e poi saranno i relativi server a comunicare tra di loro prima di recapitare il messaggio al terminale di competenza.

~

La \figurename ~\ref{fig:lan} mostra il funzionamento corretto di WhatsApp.  

~

\begin{figure}[!ht]
\centering
\includegraphics[scale = 0.8]{true_lan.png}
\caption{Funzionamento corretto di WhatsApp}
\label{fig:lan}
\end{figure}



\chapter{Conclusioni}
\end{document}
