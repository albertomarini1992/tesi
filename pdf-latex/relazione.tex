\documentclass[a4paper,11pt]{book}
\usepackage [T1]{fontenc}
\usepackage [utf8]{inputenc}
\usepackage [italian]{babel}
\usepackage {graphicx}
\usepackage {frontespizio}

\begin{document}


\begin{frontespizio}
\Logo{logo.jpg}
\Universita{Verona}
\Facolta{Scienze e Ingegneria}
\Corso[Laurea]{Informatica}
\Titolo{Whatsapp Messenger}
\Relatore{Prof. Damiano Carra}
\Candidato[VR359129]{Alberto Marini}
\Annoaccademico{2013 - 2014}
\end{frontespizio}


\tableofcontents

\chapter{Introduzione}


\chapter{Strumenti utilizzati}

\section{Whatsapp Messenger}
\subsection{Cos'è}
WhatsApp Messenger è un'app di messaggistica mobile multi-piattaforma che consente di scambiarsi messaggi coi propri contatti senza dover pagare gli SMS. WhatsApp Messenger è disponibile per iPhone, BlackBerry, Android, Windows Phone e Nokia. La risposta è sì: tutti questi telefoni possono scambiarsi messaggi gli uni gli altri! Dato che WhatsApp Messenger si serve dello stesso piano dati Internet usato per le e-mail e la navigazione web, non vi sono costi aggiuntivi per mandare messaggi e restare in contatto coi propri amici.

Oltre alla messaggistica di base gli utenti di WhatsApp possono creare gruppi, scambiarsi messaggi illimitati, video e messaggi audio multimediali.

L'11 aprile 2014 e' arrivato il via libera all'acquisizione di WhatsApp da parte di Facebook dalla Federal Trade Commission (Ftc), l'ente governativo americano per la protezione dei consumatori.

~

\subsection{Come funziona}
Come funzionaCome funzionaCome funzionaCome funzionaCome funzionaCome funzionaCome funzionaCome funzionaCome funzionaCome funzionaCome funzionaCome funzionaCome funzionaCome funzionaCome funzionaCome funzionaCome funzionaCome funzionaCome funzionaCome funzionaCome funzionaCome 

Please see Figure ~\ref{fig:rete} for a prototype yada yada yada

funzionaCome funzionaCome funzionaCome funzionaCome funzionaCome funzionaCome funzionaCome funzionaCome funzionaCome 
\begin{figure}[h!t]
\centering
\includegraphics[scale = 0.7]{rete.png}
\caption{Funzionamento Whatsapp}
\end{figure}


\clearpage

\section{Wireshark}
\subsection{Cos'è}
Wireshark è un analizzatore di rete. It lets you interactively browse packet data from a live network or from a previously saved capture file. Inizialmente, il formato dei file catturati da Wireshark era il formato libpcap, che è il formato usato da tcpdump ed altri tools.

~

Non è necessario dire a Wireshark  il formato dei file da leggere; lo determina automaticamente. Wireshark is also capable of reading any of these file formats if they are compressed using gzip. Wireshark recognizes this directly from the file; the '.gz' extension is not required for this purpose.

A differenza di altri analizzatori di protocolli, la finestra di Wireshark mostra 3 viste dei pacchetti. It shows a summary line, briefly describing what the packet is. A packet details display is shown, allowing you to drill down to exact protocol or field that you interested in. Finally, a hex dump shows you exactly what the packet looks like when it goes over the wire.

In aggiunta, Wireshark ha alcune features che lo rendono unico. Può assemblare tutti i pacchetti in una comversazione TCP e visualizzare i dati ASCII (o EBCDIC, o hex) in questa conversazione. Display filters in Wireshark are very powerful; more fields are filterable in Wireshark than in other protocol analyzers, and the syntax you can use to create your filters is richer. As Wireshark progresses, expect more and more protocol fields to be allowed in display filters.

I pacchetti catturati sono conformi alla libreria pcap. I filtri applicabili ai pacchetti seguono le regole della libreria pcap. This syntax is different from the display filter syntax.

~
Wireshark and TShark share a powerful filter engine that helps remove the noise from a packet trace and lets you see only the packets that interest you. If a packet meets the requirements expressed in your filter, then it is displayed in the list of packets. Display filters let you compare the fields within a protocol against a specific value, compare fields against fields, and check the existence of specified fields or protocols.



\section{Whois}
Whois � un servizio utilizzabile da shell di Ubuntu che permette di visualizzare informazioni di un determinato indirizzo IP.
In particolare, 





\section{Cloud Monitor}
Cloud Monitor � un'azienda leader nel settore del monitoraggio delle prestazioni di siti e applicazioni Web. Verifica le prestazioni di siti e server grazie a 95 stazioni di monitoraggio disposte in 48 paesi. 
Dato un indirizzo IP, effettua, attraverso le 95 stazioni, ping verso quell'indirizzo registrando, in caso di ping eseguito con successo, RTT minimo, RTT medio e RTT massimo (RTT - Round Trip Time, tempo impiegato da un pacchetto di dimensione trascurabile per viaggiare da un computer ad un altro e tornare indietro). 


\chapter{Misurazioni}
Misurazioni + grafici
\chapter{Conclusioni}
\end{document}
